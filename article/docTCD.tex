\documentclass[
	% -- opções da classe memoir --
	article,			% indica que é um artigo acadêmico
	12pt,				% tamanho da fonte
	oneside,			% para impressão apenas no recto. Oposto a twoside
	a4paper,            % tamanho do papel. 
	twocolumn,
	% -- opções do pacote babel --
	english,			% idioma adicional para hifenização
	brazil,				% o último idioma é o principal do documento
	sumario=tradicional
	]{abntex2}


% ---
% PACOTES
% ---

% ---
% Pacotes fundamentais 
% ---
\usepackage{lmodern}			% Usa a fonte Latin Modern
\usepackage[T1]{fontenc}		% Selecao de codigos de fonte.
\usepackage[utf8]{inputenc}		% Codificacao do documento (conversão automática dos acentos)
\usepackage{indentfirst}		% Indenta o primeiro parágrafo de cada seção.
\usepackage{nomencl} 			% Lista de simbolos
\usepackage{color}				% Controle das cores
\usepackage{graphicx}			% Inclusão de gráficos
\usepackage{microtype} 			% para melhorias de justificação
% ---
		
% ---
% Pacotes adicionais, usados apenas no âmbito do Modelo Canônico do abnteX2
% ---
\usepackage{lipsum}				% para geração de dummy text
% ---
		
% ---
% Pacotes de citações
% ---
\usepackage[brazilian,hyperpageref]{backref}	 % Paginas com as citações na bibl
\usepackage[alf]{abntex2cite}	% Citações padrão ABNT
% ---

% ---
% Configurações do pacote backref
% Usado sem a opção hyperpageref de backref
\renewcommand{\backrefpagesname}{Citado na(s) página(s):~}
% Texto padrão antes do número das páginas
\renewcommand{\backref}{}
% Define os textos da citação
\renewcommand*{\backrefalt}[4]{
	\ifcase #1 %
		Nenhuma citação no texto.%
	\or
		Citado na página #2.%
	\else
		Citado #1 vezes nas páginas #2.%
	\fi}%
% ---

% ---
% Informações de dados para CAPA e FOLHA DE ROSTO
% ---
\titulo{Destravando uma Trava Elétrica Utilizando Impressão Digital do Smartphone via PIC16F873A  }
\autor{Guilherme Camargo de Oliveira \thanks{gui551@hotmail.com}\\
      }
\local{Brasil}
\data{\today}
% ---

% ---
% Configurações de aparência do PDF final

% alterando o aspecto da cor azul
\definecolor{blue}{RGB}{41,5,195}

% informações do PDF
\makeatletter
\hypersetup{
     	%pagebackref=true,
		pdftitle={\@title}, 
		pdfauthor={\@author},
    	pdfsubject={Modelo de artigo científico},
	    pdfcreator={LaTeX with abnTeX2},
		pdfkeywords={abnt}{latex}{abntex}{abntex2}{atigo científico}, 
		colorlinks=true,       		% false: boxed links; true: colored links
    	linkcolor=blue,          	% color of internal links
    	citecolor=blue,        		% color of links to bibliography
    	filecolor=magenta,      		% color of file links
		urlcolor=blue,
		bookmarksdepth=4
}
\makeatother
% --- 

% ---
% compila o indice
% ---
\makeindex
% ---

% ---
% Altera as margens padrões
% ---
\setlrmarginsandblock{3cm}{2cm}{*}
\setulmarginsandblock{3cm}{2cm}{*}
\checkandfixthelayout
% ---

% --- 
% Espaçamentos entre linhas e parágrafos 
% --- 

% O tamanho do parágrafo é dado por:
\setlength{\parindent}{1.2cm}

% Controle do espaçamento entre um parágrafo e outro:
\setlength{\parskip}{0.2cm}  % tente também \onelineskip

% Espaçamento simples
\SingleSpacing

% ----
% Início do documento
% ----
\begin{document}

% Seleciona o idioma do documento (conforme pacotes do babel)
%\selectlanguage{english}
\selectlanguage{brazil}

% Retira espaço extra obsoleto entre as frases.
\frenchspacing 

% ----------------------------------------------------------
% ELEMENTOS PRÉ-TEXTUAIS
% ----------------------------------------------------------

%---
 \twocolumn[    		% INICIO DE ARTIGO EM DUAS COLUNAS
%
%---
% página de titulo
\maketitle

% resumo em português
\begin{resumoumacoluna}
Este projeto foi desenvolvido como Trabalho de Conclusão de Disciplina (TCD) para a matéria de Microcontroladores da Universidade Estadual Paulista (UNESP) - Bauru, em 2018. Consiste de um aplicativo Ionic que se comunica com um módulo Bluetooth HC-06 para controlar uma trava elétrica de 12v e alternar um Led RGB entre as cores vermelho, verde e azul.
 
 \vspace{\onelineskip}
 
 \noindent
 \textbf{Palavras-chave}: PIC16F873A. \emph{Internet Of Things}. TCD.
 
 \vspace{\onelineskip}
 
\end{resumoumacoluna}

 ]  				% FIM DE ARTIGO EM DUAS COLUNAS
% ---

% ----------------------------------------------------------
% ELEMENTOS TEXTUAIS
% ----------------------------------------------------------
\textual

% ----------------------------------------------------------
% Introdução
% ----------------------------------------------------------
\section*{Introdução}
\addcontentsline{toc}{section}{Introdução}
Vivemos em uma época onde a Internet das Coisas esta presente, muitos querem automatizar sua casa. Quando falamos de revolução tecnológica, a noção da Internet das Coisas, \emph{Internet of Things - IoT}, é um dos assuntos principais.  E não é difícil entender o porquê. Suas possibilidades são inúmeras, como transformar nossa relação com a tecnologia, mudar o modo como interagimos com o mundo e até mais, o modo como o mundo interage conosco.

Internet das Coisas esta relacionado com o modo como as coisas estão conectadas e se comunicam entre si e com o usuário, através de sensores inteligentes e softwares que transmitem dados para uma rede. Como se fosse um grande sistema nervoso que possibilita a troca de informações entre dois ou mais pontos.

Ela possibilita inúmeras oportunidades e conexões, muitas das quais não conseguimos imaginar nem entender completamente seu impacto nos dias de hoje.

Não é difícil de perceber por que esse assusto tem sido tão comentado atualmente, ele certamente abre portas para muitas oportunidades, mas também alguns desafios.

Nesse processo , a ideia do projeto lida com as travas elétricas , onde podemos utilizar nossos dispositivos como chave. Atualmente, encontramos muito dispositivos no mercado para abrir travas, necessitando gastar dinheiro para compra-los. A proposta desse projeto é utilizar os recursos de segurança do smartphone como a impressão digital para acionar uma trava, e portanto diminuindo nossos gastos e aumentando a segurança.

A questão de segurança dos celulares evoluiu bastante, dispositivos como impressão digital, leitura facial e leitura das íris, nos permite ter cartões de crédito digitais e ter controle de acessos restritos.


% ----------------------------------------------------------
% Seção de explicações
% ----------------------------------------------------------
\section{Recursos}

\subsection{Material Utilizado}
Para o circuito, vamos utilizar o seguinte material:

\begin{itemize}
    \item 1x Mini trava elétrica Solenoide 12V
    \item 1x Módulo Relé 5V 1 Canal
    \item 1x Fonte de 12V dc
    \item 1x Resistor 5.1Kohm
    \item 1x Capacitor 22pF
    \item 1x Fonte de 5v para o pic
    \item 1x PIC16f873A
    \item 1x Módulo Bluetooth HC-06
    \item 1x Smarthphone com Android
    \item Framework IONIC
\end{itemize}

\subsection{Trava Elétrica Solenóide}
Como o PIC trabalha em seus pinos apenas com 5V e uma corrente relativamente baixa (20mA) para acionamento de dispositivos, um relé pode ser usado para acionar dispositivos que requerem mais de 5V e correntes maiores como é o caso da trava elétrica solenoide. A trava elétrica solenóide precisa de 12V e 600mA para funcionar corretamente. A imagem da \autoref{fig:rele-trava} nos mostra o relé e a trava elétrica utilizada.

A trava elétrica solenoide funciona aplicando uma tensão de 12V em seus terminais. Então o pino da trava é recolhido, mantendo-se na posição enquanto a tensão estiver sendo aplicada. Quando não há tensão, o pino volta ao seu estado normal. A trava foi testada também com uma fonte de 9V e notou-se apenas que a força de recolhimento foi reduzida, mas não impediu o funcionamento.

\begin{figure}
    \centering
    \begin{center}
        \includegraphics[scale=0.05]{img/Rele-Trava.jpg}
    \end{center}
    
    \caption{Relé e Trava Elétrica}
    \legend{Acima podemos ver o Relé que esta identificado pela cor azul do lado esquerdo da figura. A Trava econtra-se ao lado direito da figura representado pela cor cinza}
    \label{fig:rele-trava}
\end{figure}


\subsection{Framework IONIC}

O Ionic é um framework open source para desenvolvimento de aplicativos móveis multiplataforma. Para isso, possibilita a implementação do app utilizando tecnologias comumente empregadas na construção do Front-end de soluções web: HTML, CSS e JavaScript. No entanto, como diferencial em relação ao frameworkque adota como base, o Apache Cordova, traz recursos que simplificam ainda mais o desenvolvimento e dão ao app um aspecto mais profissional.

Esses diferenciais estão relacionados ao conjunto de componentes visuais que podemos utilizar para construção do front-end da solução, assim como ao fato do Ionic trazer consigo outra linguagem e framework para prover uma solução de mais alto nível em termos de código e, consequentemente, projeto. Estamos falando do TypeScript e do Angular.

Assim, temos a inovação do Cordova, Orientação a Objetos em JavaScript, bem como as propostas que o Angular implementa em um só framework para construir apps mobile híbridas. Tudo isso faz do Ionic a principal opção quando o objetivo é criar apps mobile multiplataforma.


\section{Funcionamento do Aplicativo Mobile}

Para esse projeto foi utilizado o protocolo USART com conexão bluetooth. Na ~\autoref{USART} encontramos mais informações sobre este protocolo. 

O aplicativo consiste em três funções: 
\begin{enumerate}
    \item Unlock
    \item Party
    \item Connect Bluetooth
\end{enumerate}

\subsection{Unlock}
    Tem como funcionalidade liberar a trava por aproximadamente 10 segundos. O led
RGB muda para a cor verde indicando que foi destravado.

Internamente, o aplicativo envia o byte 0x01 por bluetooth. A responsabilidade do PIC é tratar o byte recebido e chamar a subrotina opcao1 que corresponde ao byte enviado por esse botão.

\subsection{Party}
Ao clicar nesse botão, o Led RGB conectado ao circuito ira começar a piscar, alternando entre as cores: vermelha, azul, verde. E aparecera uma janela, caso queira que o Led pare de piscar.

Basicamente, a aplicação envia o byte 0x02 por bluetooth ao PIC que chama a subrotina opcao2 para iniciar o processo de piscar.

Quando o botão cancelar é ativado, a aplicação envia o byte 0x03 por bluetooth ao PIC que chama a subrotina opcao3 para parar de piscar o Led.
    
\subsection{Connect Bluetooth}
Parte mais importante, deve ser utilizada primeiro. Sua função é parear o bluetooth do smatphone com o módulo conectado ao PIC.

Se o emparelhamento foi um sucesso então o smartphone ira mostrar uma mensagem \emph{OK!}. Além dessa mensagem, o led do Módulo Bluetooth ira parar de piscar indicando sucesso tambem.

\begin{figure}[!h]
    \centering
    \includegraphics[scale=0.1]{img/Control.jpg}
    \caption{Tela Aplicativo Mobile}
    \legend{Screenshot real do aplicativo desenvolvido.}
    \label{fig:appMobile}
\end{figure}

\section{Discussões}

\subsection{Base digital de Biometria}
O aplicativo utiliza a biblioteca Fingerprint AIO do IONIC. 

Ela simplesmente utiliza as impressões digitais que o smartphone possui registradas. A API solicita ao Android , se a impressão digital lida no momento esta registrada. Caso esteja correta , ela retorna um \emph{callback} a aplicação.

Ou seja, caso queira adicionar mais impressões digitais, basta ir em configurações , segurança e add nova impressão digital, dentro do SO (Android/IOS) do smartphone.

\subsection{Clock}
Consideração importante , pois o PIC não tem clock interno. Portanto houve a necessidade de colocar um circuito RC ( Resistor - Capacitor ) em série para gerar o clock. Podemos utilizar de cristais como mostra o datasheet, porém não havia disponivel no momento.

Utilizando a fórmula que se encontra no datasheet, obtemos uma Frequência de oscilação de 3.8707MHz. Lembrando que usamos um resistor de 5.1k com um capacitor de 22pF.

\subsection{USART}
\label{USART}
Para a comunicação entre o aplicativo e o PIC, utilizamos o protocolo USART que se refere a Universal Synchronous Asynchronous Receiver Transmitter, significando Transmissor/Receptor Universal Síncrono e Assíncrono. 


\begin{figure}[!h]
    \centering
    \includegraphics[scale=0.05]{img/ciruito1.jpg}
    \caption{Foto do circuito pronto}
    \label{fig:circuito1}
\end{figure}

É um formato padrão para comunicação de dados de forma serial. Em forma assíncrona, dois fios são usados para transmitir dados, um em cada direção, em regime full-duplex, ou seja, totalmente bi-direcional. Para isso, cada dispositivo deve ter seu clock, e as velocidades devem ser iguais. Em forma síncrona, uma ponta é mestre e a outra escravo. Um fio é utilizado para dados, em regime half-duplex, ou seja, nos dois sentidos, mas um sentido de cada vez. O outro fio é usado para pulsos de clock emitidos pelo dispositivo mestre.


\begin{figure}[!h]
    \centering
    \includegraphics[scale=0.05]{img/circuito2.jpg}
    \caption{Foto do circuito pronto em outro angulo}
    \label{fig:circuito2}
\end{figure}

% ---
% Finaliza a parte no bookmark do PDF, para que se inicie o bookmark na raiz
% ---
\bookmarksetup{startatroot}% 
% ---

% ---
% Conclusão
% ---
\section*{Considerações finais}
\addcontentsline{toc}{section}{Considerações finais}

Podemos ver que a ideia do projeto é interessante pois poderemos aplicar na gaveta da estante, porta do armário ou em qualquer compartimento onde queiramos fazer o controle de acesso usando uma trava elétrica e um smartphone. Sem necessitar comprar dispositivos de leitor impressão digital, pois o smartphone ja traz consigo.

O código desse projeto encontra-se em \url{https://github.com/GuiCamargoX/Project-PIC16F873A.git}. Podemos observar imagens do circuito pronto nas \autoref{fig:circuito1} e \autoref{fig:circuito2}.

O objetivo da Disciplina foi alcançado. Utilizamos de Assembly para programar o PIC, tive que configurar os bits da conexão USART. Formas como fazer comandos de alto nível como loops, if e switch case foram necessarias transformar em comandos assembly.
Aprender a mexer no Oscilador RC externo para conseguir um clock estavel. Integrar Bluetooth no app mobile utilizando IONIC e
aprender a framework IONIC propriamente.


% ----------------------------------------------------------
% ELEMENTOS PÓS-TEXTUAIS
% ----------------------------------------------------------
\postextual

% ---

% ----------------------------------------------------------
% Referências bibliográficas
% ----------------------------------------------------------
\bibliography{abntex2-modelo-references}

% ----------------------------------------------------------
% Glossário
% ----------------------------------------------------------
%
% Há diversas soluções prontas para glossário em LaTeX. 
% Consulte o manual do abnTeX2 para obter sugestões.
%
%\glossary

% ----------------------------------------------------------
% Apêndices
% ----------------------------------------------------------

% ---
% Inicia os apêndices
% ---

% ----------------------------------------------------------
% Anexos
% ----------------------------------------------------------


\end{document}
